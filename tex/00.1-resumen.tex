% Resumen de la Tesis
% ---------------------------------------------------------------------------------------------------------------
\prefacesection{Resumen}

El crecimiento que ha tenido el sistema operativo Android es considerable. Existen m�s de un mill�n de aplicaciones disponibles en la tienda de Google y cada mes este n�mero se ve incrementado. Es por ello que el proceso de desarrollo de aplicaciones ha ganado vital importancia. El objetivo de esta memoria es estudiar y comparar herramientas que ayuden a mejorar el desarrollo de aplicaciones Android de tal manera de proveer a los desarrolladores una gu�a pr�ctica que les permita tomar mejores decisiones en el transcurso de un proyecto.

Lorem ipsum ad his scripta blandit partiendo, eum fastidii accumsan euripidis in, eum liber hendrerit an. Qui ut wisi vocibus suscipiantur, quo dicit ridens inciderint id. Quo mundi lobortis reformidans eu, legimus senserit definiebas an eos. Eu sit tincidunt incorrupte definitionem, vis mutat affert percipit cu, eirmod consectetuer signiferumque eu per. In usu latine equidem dolores. Quo no falli viris intellegam, ut fugit veritus placerat per.


\textbf{Palabras Claves:} 7 conceptos m�ximos separados por coma.
% ---------------------------------------------------------------------------------------------------------------