% Resumen de la Tesis
% ---------------------------------------------------------------------------------------------------------------
\prefacesection{Resumen}

El crecimiento que ha tenido el sistema operativo Android en el mercado de las tecnolog�as m�viles es considerable. Existe m�s de un mill�n de aplicaciones disponibles en la tienda de Google y cada mes este n�mero se ve incrementado. Es por ello que el proceso de desarrollo de aplicaciones para Android ha ganado vital importancia. 

El objetivo de esta memoria es estudiar y comparar herramientas que ayuden a mejorar el desarrollo de aplicaciones Android, entregando a los desarrolladores una gu�a pr�ctica que les permita tomar mejores decisiones durante el transcurso de un proyecto.

Los problemas m�s comunes que se enfrentan al desarrollar para Android est�n relacionados con la fragmentaci�n, tanto a nivel de software como de hardware. Por otro lado est� la distribuci�n de versiones betas antes de una publicaci�n oficial y el manejo de los ca�das. Para solucionar y mitigar estos problemas se han estudiado y comparado herramientas clasificadas en las siguientes categor�as: testing, distribuci�n de versiones y reporte de crashes. En la comparaci�n se consideran par�metros que ayudan a los desarrolladores a discernir sobre qu� usar, teniendo en cuenta su costo, usabilidad, madurez, documentaci�n, soporte para m�ltiples plataformas, entre otras. 

Para validar las caracter�sticas de las herramientas estudiadas, se implementaron las que m�s se adaptaban a las necesidades de una aplicaci�n desarrollada por el equipo de Android de Seahorse, del cu�l formo parte.

\textbf{Palabras Claves:} Android, Aplicaciones m�viles, Testing, Distribuci�n de versiones, Reporte de crashes.
% ---------------------------------------------------------------------------------------------------------------