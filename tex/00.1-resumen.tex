% Resumen de la Tesis
% ---------------------------------------------------------------------------------------------------------------
\prefacesection{Resumen}

El crecimiento que ha tenido el sistema operativo Android es considerable. Existen m�s de un mill�n de aplicaciones disponibles en la tienda de Google y cada mes este n�mero se ve incrementado. Es por ello que el proceso de desarrollo de aplicaciones para Android ha ganado vital importancia. 

El objetivo de esta memoria es estudiar y comparar herramientas que ayuden a mejorar el desarrollo de aplicaciones Android de tal manera de proveer a los desarrolladores una gu�a pr�ctica que les permita tomar mejores decisiones en el transcurso de un proyecto.

Las herramientas a estudiar se pueden clasificar en 3 tipos: testing, distribuci�n de versiones y reporte de crashes. Para su comparaci�n se considerar�n par�metros que ayuden a los desarrolladores a discernir sobre que herramienta usar, como su precio, complejidad al momento de implementar, si es que es multiplataforma, entre otras. Por �ltimo se implementar�n las herramientas m�s destacadas en un entorno real de desarrollo.


\textbf{Palabras Claves:} Android, Aplicaciones m�viles, Testing, Distribuci�n de versiones, Reporte de crashes.
% ---------------------------------------------------------------------------------------------------------------