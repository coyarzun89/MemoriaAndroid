% Cap�tulo 5: Conclusiones y Trabajo Futuro
% ---------------------------------------------------------------------------------------------------------------
\chapter{Conclusiones}
\label{ch:conc}
A trav�s de este trabajo, se han estudiado y comparado diversas herramientas que ayudan a mejorar la calidad de las aplicaciones desarrolladas para Android. Esto se ha llevado a cabo clasificando las distintas herramientas en tres categorias principales: Testing, Distribuci�n de versiones y Reportes de Crashes.  

En cada una de las categor�as se ha realizado un an�lisis comparativo, a trav�s del cual se han obtenido las ventajas y desventajas de cada una de las herramientas. En algunos casos ha sido necesario subdividir las categor�as para que las comparaciones arrojen resultados m�s �tiles. Esto permite que los desarrolladores cuenten con informaci�n relevante que les ayude a tomar mejores decisiones al momento de tener que implementar una de estas herramientas. 

En base a las caracter�sticas de cada herramienta, se han implementado algunas de estas en una aplicaci�n que est� disponible en la tienda oficial de Google. Cabe mencionar que las herramientas implemetadas se ajustaban m�s a las necesidades de Seahorse, ya que es posible que otros desarrolladores, en base a las ventajas y desventajas presentadas, implementen otro conjunto de herramientas, que se ajuste m�s al contexto en que trabajan.

Documentacion Obsoleta
eclipse a android studio

...



\section{Trabajo Futuro}

% ---------------------------------------------------------------------------------------------------------------
