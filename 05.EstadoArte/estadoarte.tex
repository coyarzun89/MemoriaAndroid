En este cap�tulo se dar� a conocer como el desarrollo de aplicaciones m�viles en Android ha ido evolucionando con el tiempo, suscitado un gran inter�s en la �ltima d�cada, debido principalmente a su crecimiento exponencial, el cu�l parece no tener l�mite. Adem�s se hablar� sobre los problemas m�s comunes al momento de comenzar a desarrollar una aplicaci�n para Android

\section{Inicios de Android}
Android, Inc. fue fundada en Palo Alto, California en Octubre del 2003 por Andy Rubin, Rich Miner, Nick Sears and Chris White. Su objetivo era desarrollar dispositivos m�viles m�s inteligentes, que estuvieran m�s enfocados en la localizaci�n del due�o y en distintas preferencias.

Google compr� a Android Inc. el 17 de Agosto del 2005. Poco se sab�a sobre esta compa��a para ese entonces ya que estuvo funcionando de forma secreta, sin dar a conocer muchos detalles sobre lo que desarrollaban. Muchos asum�an que Google estaba planeando entrar al mercado de dispositivos m�viles. De ahi en adelante los esfuerzos de Google se enfocaron en conversaciones con fabricantes y carriers, con la promesa de proveer de un sistema flexible y actualizable. 

Sin embargo, la aparici�n del iPhone el 9 de Enero del 2007 tuvo un efecto disruptivo en el desarrollo de Android. Hasta el momento se contaba con un prototipo, el cu�l se acercaba m�s a lo que podr�a ser un tel�fono BlackBerry, sin pantalla t�ctil y con un teclado f�sico. Por lo que se comenz� inmediatamente un trabajo de reingenier�a del sistema operativa y del prototipo para que fuese capaz de competir con el iPhone.

El 6 de Noviembre del 2007 fue fundada la Open Handset Alliance, una alianza comercial liderada por Google con compa��as tecnol�gicas como HTC, Sony y Samsung, operadores de carriers como Nextel y T-Mobile y fabricantes de chips, con el objetivo de desarrollar est�ndares abiertos para dispositivos m�viles. El primer smartphone disponible que funcionaba sobre Android fue el HTC Dream, lanzado el 22 de Octubre del 2008.


\section{Problemas al desarrollar en Android}
Lorem ipsum ad his scripta blandit partiendo, eum fastidii accumsan euripidis in, eum liber hendrerit an. Qui ut wisi vocibus suscipiantur, quo dicit ridens inciderint id. Quo mundi lobortis reformidans eu, legimus senserit definiebas an eos. Eu sit tincidunt incorrupte definitionem, vis mutat affert percipit cu, eirmod consectetuer signiferumque eu per. In usu latine equidem dolores. Quo no falli viris intellegam, ut fugit veritus placerat per.


\subsection{Fragmentaci�n a nivel de software}
Aqu� se habla de que existen muchos sistemas operativos de Android activos hoy en d�a.
Lorem ipsum ad his scripta blandit partiendo, eum fastidii accumsan euripidis in, eum liber hendrerit an. Qui ut wisi vocibus suscipiantur, quo dicit ridens inciderint id. Quo mundi lobortis reformidans eu, legimus senserit definiebas an eos. Eu sit tincidunt incorrupte definitionem, vis mutat affert percipit cu, eirmod consectetuer signiferumque eu per. In usu latine equidem dolores. Quo no falli viris intellegam, ut fugit veritus placerat per.

\subsection{Fragmentaci�n a nivel de hardware}
Aqu� existe fragmentaci�n en muchos sentidos, por un lado tenemos los tama�os de pantalla distintos, las resoluciones de pantalla distintas. Por otro lado hay que considerar que cada dispositivo tiene una cantidad de espacio y memoria distintos, como tambi�n que algunos pueden tener un teclado f�sico, puede que no tengan c�mara, etc. Por �ltimo, muchas veces los fabricantes como Samsung, hacen cambios en el sistema operativo, cambiando cosas nativas, lo cu�l produce diferentes experiencias en cada dispositivo, y muchas veces genera errores que no son culpa del desarrollador.

Lorem ipsum ad his scripta blandit partiendo, eum fastidii accumsan euripidis in, eum liber hendrerit an. Qui ut wisi vocibus suscipiantur, quo dicit ridens inciderint id. Quo mundi lobortis reformidans eu, legimus senserit definiebas an eos. Eu sit tincidunt incorrupte definitionem, vis mutat affert percipit cu, eirmod consectetuer signiferumque eu per. In usu latine equidem dolores. Quo no falli viris intellegam, ut fugit veritus placerat per.

\subsection{Otros problemas}
Lorem ipsum ad his scripta blandit partiendo, eum fastidii accumsan euripidis in, eum liber hendrerit an. Qui ut wisi vocibus suscipiantur, quo dicit ridens inciderint id. Quo mundi lobortis reformidans eu, legimus senserit definiebas an eos. Eu sit tincidunt incorrupte definitionem, vis mutat affert percipit cu, eirmod consectetuer signiferumque eu per. In usu latine equidem dolores. Quo no falli viris intellegam, ut fugit veritus placerat per.


   

